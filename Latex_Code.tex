\documentclass[conference]{IEEEtran}
\IEEEoverridecommandlockouts
% The preceding line is only needed to identify funding in the first footnote. If that is unneeded, please comment it out.
\usepackage{cite}
\newcommand{\imagepath}{C:/Users/Lenovo/Desktop/SEM 6/BIG DATA/BIG_DATA_COURSEWORK/Images for Latex/}
\usepackage{comment}
%\documentclass{article}
\usepackage{float}
\usepackage{graphicx}
\usepackage{amsmath,amssymb,amsfonts}
\usepackage{algorithmic}
\usepackage{graphicx}
\usepackage{tabularx}
\usepackage{xcolor} \pagecolor[rgb]{0,0,0} \color[rgb]{1,1,1} % esle pdf viewer lai dark mode bvanako ho 
\usepackage{textcomp}
\usepackage{xcolor}
\usepackage{enumitem}
%\setlist[itemize]{label=--}



\def\BibTeX{{\rm B\kern-.05em{\sc i\kern-.025em b}\kern-.08em
    T\kern-.1667em\lower.7ex\hbox{E}\kern-.125emX}}
\begin{document}

\title{Implementing Big Data Analytics for tourist budget planning in New York City by matching tourist preferences through predictive modeling of Airbnb rental prices\\
{\footnotesize \textsuperscript{*}}
%\thanks{Identify applicable funding agency here. If none, delete this.}
}

\author{\IEEEauthorblockN{Sujan Neupane}
\IEEEauthorblockA{\textit{Herald College Kathmandu} \\
\textit{University of Wolverhampton}\\
WLV ID: 2058939 \\
Kathmandu, Nepal \\
neupanesujan420@gmail.com}
\and
\IEEEauthorblockN{Nischay Shakya}
\IEEEauthorblockA{\textit{Herald College Kathmandu} \\
\textit{University of Wolverhampton}\\
WLV ID: 2059157\\
Kathmandu, Nepal \\
nischayshakya@gmail.com
}
}

\maketitle
\begin{abstract}
This paper proposes a data-driven approach to help tourists with budget planning for Airbnb rentals 
in New York City. The popularity of Airbnb has increased significantly for tourists looking for affordable
 and convenient accommodations. This paper proposes an approach for a predictive model to match tourist preferences 
 with rental price ranges, allowing for more effective budget planning. NYC listings have been analyzed and trained under various models to 
 predict rental prices based on multiple features. Different supervised learning models have been  implemented. 
\end{abstract}

\begin{IEEEkeywords}
Airbnb Price Prediction, Big Data Analytics, Logistic Regression, Random Forest, XGBoost, LightGBM, Classification, Price Binning and Categorization
\end{IEEEkeywords}

%\vspace{0\baselineskip} % Add one line of space
% ############################# ------------------------------------------------- BACKGROUND OF STUDY
\section{Background of the Study}

Today’s modern and sophisticated technology has diversified the tourist experience vastly.
 The rise of the internet has allowed for better communication and for the rise of online service-providing marketplaces such as Airbnb. Airbnb is an online 
 platform that allows hosts to rent their living spaces to travelers who are looking for a place to stay. The living space could range from their homes and 
 apartments to hotels, cottages, and penthouses [cite: https://www.airbnb.com/help/article/2503]. With these advanced approaches replacing the traditional ways
  of traveling and lodging, the options for tourists to find the perfect place to stay has increased. The search for the most affordable and budget-friendly
places in cities like New York as per their requirements has been made possible by Airbnb. However, finding such rentals is a frustrating and 
time-consuming task which can often lead to disappointment.

% ############################# ------------------------------------------------- BACKGROUND OF STUDY - Problem Statement
\subsection*{\centering A. Problem Statement}

The IEEEtran class file is used to format your paper and style the text. All margins, 
column widths, line spaces, and text fonts are prescribed; please do not 
alter them. You may note peculiarities. For example, the head margin
measures proportionately more than is customary. This measurement 
and others are deliberate, using specifications that anticipate your paper 
as one part of the entire proceedings, and not as an independent document. 
Please do not revise any of the current designations.

% ############################# ------------------------------------------------- BACKGROUND OF STUDY - Aims/Objectives
\subsection*{\centering B. Aims/Objectives}
SThe primary aim of the project is to assist tourists wanting to visit New York City with efficient budget planning through price prediction for Airbnb rentals. 
The aim of this study is to propose a data-driven approach to help tourists with budget planning for Airbnb rentals in New York City. We seek to develop a predictive model that can match tourist preferences with rental prices, allowing for more effective budget planning. This approach can help tourists make informed decisions about which rentals to choose based on their preferences and budget, while also helping hosts set appropriate rental prices.


% ############################# ------------------------------------------------- BACKGROUND OF STUDY - Contributions of the work connected with Methodology
\subsection*{\centering C. Contributions of the work connected with Methodology}
The contributions of this work are connected with the methodology used to develop the predictive model. Our approach combines both supervised and unsupervised learning techniques, which allows us to accurately predict rental prices based on key features and tourist preferences. Specifically, our contributions include:
Data preprocessing: We preprocess the Airbnb dataset from NYC listings to ensure that it is suitable for machine learning analysis.
Feature selection: We use feature selection techniques to identify the most important features that are relevant to predicting rental prices.
Model selection: We compare several machine learning algorithms to determine which one performs best at predicting rental prices.
Model evaluation: We evaluate the performance of the selected model by comparing its predictions to actual rental prices



% ############################# ------------------------------------------------- BACKGROUND OF STUDY - Organization of Report
\subsection*{\centering D. Organization of the Report}
The report is organized as follows. The next section discusses related work on data-driven approaches to tourism planning and management. The methodology section describes the data preprocessing, feature selection, and machine learning algorithms used in this study. The results and discussion section presents our findings and provides insights into the predictive model's performance. Finally, the conclusion section summarizes the contributions of this work and discusses future research directions.


\section{Related Works}
% ############################# ------------------------------------------------- RELATED WORKS
The tourism industry has overturned its traditional business models to new ways with the growth of sharing platforms such as Airbnb that provide peer-to-peer (P2P) accommodation. There have been both positive and negative impacts on the socio-cultural, economic, and environmental areas of tourism and hospitality. . The increasing popularity of Airbnb also makes it a difficult task for the rental owners to place a reasonable price on their rental properties that is fair but also profitable. Similarly, the customers also need to know if the offered price is reasonable, in their budget, and covers all their requirements. The paper aims to forecast Airbnb prices across New York City using various machine learning approaches and predictive modeling techniques to find the dependent and independent variables along with the correlations between them, despite the insignificant amount of knowledge of the property.\newline

Airbnb has expanded from a small start-up business in the field of short-term renting to a global powerhouse. With such growth, also came challenges and concerns, mostly about the pricing of the rentals. Airbnb allows its hosts to take full responsibility for setting the expected price of their items independently. This task is reasonably difficult when considering many different factors, especially with the pricing models available in the market being either not free or inaccurate. The authors concluded that the hosts can’t keep the same price for all dates to ensure appropriate prices that compete with the current market. It was clear from past research that quantitative pricing was not enough to train the model properly and make the models robust enough to predict Airbnb prices. To fix the problems, careful consideration should be given to exploratory data analysis to first make datasets more suitable. The paper suggests the use of cross-validation and random search to tune each parameter in each model. XGboost was selected as the best model just like for Airbnb price prediction in Amsterdam with an R2 score of 0.6321.\newline

Similarly, the paper by which predicted the prices on two New York City datasets from 2019 and 2022, also concluded XGBoost to be the best model for predicting Airbnb rental prices. They compared various machine learning techniques namely K-Nearest Neighbors, Support Vector Machines, Decision Tree, Random Forests, and XGBoost. The results of the studies suggest that machine learning models can be applied to accurately predict Airbnb rental prices. The major factor for successful bookings as signifies is proper pricing. The paper focuses on one of the main markets for Airbnb, Beijing, and uses XGBoost and a three layers Neural Network model to build price-prediction models. The results showed the former performing better than the latter. The author also briefly mentioned the impact of the COVID-19 pandemic on the Beijing Airbnb market. The accuracy of the model is proved to be significantly influenced by outliers as well. Moreover, hosts are provided with recommendations about how to increase prices by exploring the influence on price by adding important amenities like Internet, TVs, Elevators, and services like cancellation policy and instant booking. \newline

The deficiencies that price predicting models face can mainly be summed up to, (1) the feature attributes being poor; (2) the lack of fundamental research on the sector of correlation between the point of interest (POI) around the house and the rental price; (3) the lack of depth on rental text information research. A multi-source information embedding (MSIE) model has been proposed to address the above challenges that combine statistical, text, and spatial features to predict the rental prices of Beijing and Shanghai by forming a multi-source rental representation. The dynamic nature of the tourism industry demands constant upgrades on even the most powerful algorithms used for rental price prediction such as XGBoost, Support Vector Regression, and Regression Tree. proposed a three-layered deep neural network with two outputs i.e. the minimum and maximum amount that a host could charge based on the rental information. The proposed framework was built using the Rio de Janeiro Airbnb dataset and had an accuracy of 74.43\% and outperformed the existing models with an R2 value of 0.8104. The paper also identified the top factors that should be included in rentals to optimize profits. \newline

In a similar approach by, it has been shown that multi-modality data performs better in price forecasting in comparison to single-type data. In the research, numeric data, text data, and map data have been incorporated into a framework for price prediction by combining customer reviews, house features, and geographical data as input for machine learning algorithms and natural language processing framework. The paper has also concluded Principle Component Analysis (PCA) as an effective approach for Exploratory Data Analysis (EDA) for the New York City dataset. The paper is in favor of exotic algorithms such as DNN and XGBoost to perform better than linear models such as Linear Regression and Support Vector Regression. Alternatively, the paper by analyzed that Machine Learning algorithms such as Decision tree and KNN are the best fit for the data with accuracies 0.94 and 0.91 over models like Naïve Bayes and Logistic regression and ARIMA model is the best fit for predicting price listing. Data wrangling and hypothesis testing were also used and the model successfully predicted the price of Airbnb listings in New York City for the forthcoming years. \newline



\section{Methodology}
% ############################# ------------------------------------------------- Methodology
  

\begin{figure}[H]
  \centering
  \includegraphics[width=0.4\textwidth]{\imagepath/BLOCK_DIAGRAM_Activity_Diagram.jpeg}
  \caption{Project methodology}
  \label{fig:your_label}
\end{figure}


% ############################# ------------------------------------------------- Methodology = NYC Airbnb Dataset Retrieval
\subsection*{\centering A. NYC Airbnb Dataset Retrieval}
The 2022 Airbnb dataset for New York City was retrieved from the Inside Airbnb website. The dataset consists of multiple files, among which the file listings.csv has been utilized in this study. The dataset contains rental information for five boroughs in NYC where Airbnb rentals are operating actively. The selected file has 74 columns and 41533 rows. The listings.csv file was uploaded to MongoDB and retrieved as a pandas data frame in Jupyter Notebook using the PyMongo module for further processing.

% ############################# ------------------------------------------------- Methodology = Data Exploration and Visualization
\subsection*{\centering B. Data Exploration and Visualization}
Not all of the 74 columns in the dataset were useful. Each column was explored one by one, and columns like id, listings\_url, scrape\_id, etc were dropped as they don’t contain any relevant information. Similarly, the columns with a high percentage of null values were also dropped. The dataset had a variety of data types like an object, float, integer, etc. The selected columns underwent further data visualization, exploration, and cleaning. The following barplot represents the percentage of null values in columns. 


\begin{figure}[H]
  \centering
  \includegraphics[width=0.4\textwidth]{\imagepath/percentage_of_null_values.png}
  \caption{Percentage of null values}
  \label{fig:your_label}
\end{figure}

% ############################# ------------------------------------------------- Methodology =Data Cleaning
\subsection*{\centering C. Data Cleaning}
The target variable, price, was heavily skewed to the right. It made it difficult to directly analyse the relationship between the mean price and other variables. As a result, the median of the original price column and a new price column with outliers eliminated using the interquartile range were used for analytical purposes.

\begin{figure}[H]
  \centering
  \includegraphics[width=0.4\textwidth]{\imagepath/price_skewed.png}
  \caption{Skweness in the price}
  \label{fig:your_label}
\end{figure}

In several columns, some of the rows contained N/A, nan, and other special characters such as \$ and \%. To ensure uniformity, such special characters were removed, and N/A values were replaced with nan. The null values were then substituted by the mode for categorical columns if the percentage of null values was relatively low.  Some columns had f and t values that were replaced with True and False respectively. The relationship between the columns with null values and the price column was investigated first. Then, after filling in the null values, the updated column’s relationship with the price column was investigated to ensure that the original relationship was still maintained after imputing the null values. Some of the columns also had too much noise. Such columns were also dropped. 


% ############################# ------------------------------------------------- Methodology = Featrue Engineering
\subsection*{\centering D. Feature Engineering}
\subsection*{\centering i. Feature Extraction}
Some of the columns had to undergo transformations to extract important relevant information. The property\_type column had values like Private room in a condo, Entire guest suite, Private room in a religious building, etc. Such values were transformed to specific categories like Apartment, House, Hotel, Hostel, Bed and Breakfast, and Other. Similarly, the bathroom column had many unique values like  0 baths, 1 shared bath, 4.5 shared baths etc. Some rentals also offered multiple private and shared bathrooms. Few also offered half-bathrooms. 

\begin{figure}[H]
  \centering
  \includegraphics[width=0.4\textwidth]{\imagepath/bathroom_uncleaned.png}
  \caption{Variations in bathroom types and median price}
  \label{fig: your_label}
\end{figure}

Two new columns were extracted using the bathroom column. A new column for the number of available shared baths and another column for the number of private baths was created. Then, the original bathroom column was dropped. \newline

The amenities column had an excessive number of unique values. Some rentals had too many amenities as compared to other rentals. It made modeling the data difficult as each amenity in this column needed to be dummy encoded which would have resulted in too many new columns. Instead, each amenity was classified to one of following seven categories: safety, kitchen, bathroom, bedroom, entertainment, climate control and wifi. Finally, these categories were one-hot encoded to create new conlumns that contained number of corresponding amenity category present in the rentals.\newline

Several types of reviews were present in the dataset. Some reviews were for cleanliness of the rental, communication with host, location etc. Each of these review columns had values ranging from 0 to 5. Initially, such continuous values had a random association with the price column. As a result, values in each of the reviews columns were transformed into separate categories by binning all values from 0 to 5 at intervals of 0.5, yielding a total of 10 categories. The newly engineered features had a strong linear relationship with the price column. \newline

\begin{figure}[H]
  \centering
  \includegraphics[width=0.4\textwidth]{\imagepath/useless_scatterplot.png}
  \caption{Scatterplot between ratings and price}
  \label{fig: your_label}
\end{figure}

The ordinal categorical columns were label encoded. Columns such as review\_scores\_cleanliness, review\_scores\_checkin, review\_scores\_communication, review\_scores\_location, review\_scores\_value, Ratings, and instant\_bookable were label encoded, whereas neighbourhood\_group\_cleansed had no order and thus underwent one-hot/dummy encoding. \newline

Finally, the price column was engineered such that the continuous values in the price column were converted to ordered categories for performing classification task. Prices for Airbnb rentals ranging from \$0 to \$50 were assigned the Budget category. The rentals with price between \$51 to \$200 and \$201 to \$350 were categorized as Standard and premium respectively. Likewise, the rentals with price between \$301 to \$500 were classified as Luxury. Finally, the rentals with a price of more than \$501 were assigned to the Ultra-Luxury category. As a result, the dataset ended up with five classes. \newline

\begin{figure}[H]
  \centering
  \includegraphics[width=0.4\textwidth]{\imagepath/five_classes_count.png}
  \caption{Number of samples in each class}
  \label{fig: your_label}
\end{figure}


% #################################################################### Methodology - Feature selection
\subsection*{\centering ii. Feature Selection}
74 columns were effectively reduced to 34 columns after the preprocessing and data cleaning. The 34 columns were put through feature selection to pick the most significant columns for modeling purposes and eliminate those that were ineffective. Three feature selection techniques were implemented: Univariate Feature Selection, Recursive Feature Elimination, and Tree-Based Feature Selection. Each technique assigned an importance value to the features. To determine the overall importance of each feature, the average importance of all three feature selection techniques was calculated and the top 17 features were selected. \newline

% ############################# ------------------------------------------------- Methodology = Data Splitting
\subsection*{\centering E. Data Splitting}
Using sklearn's train\_test\_split function, the dataset was divided into two parts: the training set and the testing set in an 80:20 ratio. There were 33222 rows and 17 columns in the training set. Similarly, there were 8306 rows in the test set. Each model was trained using cross-validation, which automatically divided the original training set into training and validation sets in an 80:20 ratio. As a result, the original data was divided into three separate parts with a 60:20:20 ratio for training, validation, and testing. \newline


% ############################# ------------------------------------------------- Methodology =Model Training and Hyperparameter Tuning
\subsection*{\centering F. Model Training and Hyperparameter Tuning}
The following classifiers have been implemented.

\begin{itemize}

\item 	Logistic Regression
\item 	Random Forest Classifier
\item 	LightGBM Classifier
\item 	XGBoost Classifier \newline
\end{itemize}


Additionally, each classifier was tuned across a wide variety of hyperparameters using RandomizedSearchCV to further improve the model. The following table showcases the accuracy of each tuned model. The following table contains the performance of each model on test dataset measured across multiple metrics. 

% LINK https://tableconvert.com/excel-to-latex
\begin{table}[H]
    \centering
    \setlength{\tabcolsep}{2.9pt}
    \renewcommand{\arraystretch}{2.5}
    \begin{tabular}{|l|c|c|c|c|c|}
    \hline
        Model & Precision & Recall & F1 Score & AUC Score & Accuracy \\ \hline
        Logistic Regression & 0.64 & 0.67 & 0.61 & 0.839 & 0.67 \\ \hline
        XGBoost & 0.67 & 0.7 & 0.66 & 0.86 & 0.69 \\ \hline
        LightGBM & 0.68 & 0.7 & 0.66 & 0.859 & 0.69 \\ \hline
        Random Forest & \textbf{0.69} & \textbf{0.71} & \textbf{0.67} & \textbf{0.868} & \textbf{0.71} \\ \hline
    \end{tabular}
    \vspace{10pt} % adjust as needed
    \caption{Model performance metrics}
\end{table}

The Logistic Regression model performed the worst as it is not able to capture nonlinear or monotonic relationships in the data [https://ieeexplore.ieee.org/document/9207064-Nonlinear Logistic Regression Model Based On Simplex Basis Function]. Similarly, two gradient boosting models, XGBoost and LightGBM, performed better than Logistic regression by achieving test accuracy of approximately  0.69 each. The Random Forest model performed best with a  test accuracy of approximately  0.71. Each model except Logistic Regression was overfitting with the training data, which is why a wide variety of hyperparameters and regularization techniques were implemented to reduce overfitting and improve generalization on the validation dataset. \newline

% ############################# ------------------------------------------------- Methodology = Prediction
\subsection*{\centering G. Prediction}
After training and optimizing each model, predictions were made on the test dataset. The confusion Matrix and ROC Curve for each model is discussed in the results section.\newline


% ############################# ------------------------------------------------- Discussion
\section{Results and Discussion}

% ############################# ------------------------------------------------- Discussion = Experimental Setup
\subsection*{\centering A. Experimental Setup}
The dataset was obtained from Inside Airbnb New York City Listing. The dataset was imported into MongoDB database and then connected to Python using PyMongo library. The dataset was observed and as with any project involving data analytics, we started off by performing exploratory data analysis (EDA). Python libraries such as pandas were used for data handling and manipulation and matplotlib and seaborn were used for visualization.

\begin{itemize}
\item 	Python version: 3.10.8
\item 	Pymongo version: 4.3.3
\item 	Pandas version: 1.5.2
\item 	Matplotlib version: 3.6.2
\item 	Seaborn version: 0.12.1\newline
\end{itemize}
	
The project was implemented on Windows 11 using Jupyter Notebook on a Anaconda virtual environment. 

% ############################# ------------------------------------------------- Discussion = Analysis and Discussion of the Findings
\subsection*{\centering B. Analysis and Discussion of the Findings}
The original dataset consisted of 74 columns and 41533 rows. After data preprocessing and cleaning, the columns were reduced to 34. The 34 columns logically seemed relevant at first glance and hence will be checked to see the relationship with price which is the target variable.\newline

\begin{figure}[H]
  \centering
  \includegraphics[width=0.4\textwidth]{\imagepath/NYC_Boroughs_Prices.png}
  \caption{Heatmap showing median price of airbnb rentals in five New York City boroughs}
  \label{fig: your_label}
\end{figure}

First, the median prices of Airbnb rentals were compared amongst the boroughs of New York City. The above heatmap generated using latitude and longitude columns shows that the median price in Manhattan is the highest. Apart from the 5 boroughs, there is also a column dealing with the price in 223 neighborhoods in NYC. The importance of the columns is to be analyzed.\newline

\begin{figure}[H]
  \centering
  \includegraphics[width=0.4\textwidth]{\imagepath/hostresponsetime_and_medianprice.png}
  \caption{Host response time vs median price of five boroughs in NYC}
  \label{fig: your_label}
\end{figure}

The observation of the relationship between ‘Response Time for Hosts’ with ‘Median Price’ clearly showed that hosts that respond within an hour for rental inquiries have the highest median price. Additionally, the rentals with hosts’ response time being unknown had the lowest median price. It indicates that the potential tourists are not likely to choose rentals whose hosts response time is unknown, which ultimately drives the demand and price for such rentals down.\newline


\begin{figure}[H]
  \centering
  \includegraphics[width=0.4\textwidth]{\imagepath/number_of_Rentals_and_median_prices_of_broughs.png}
  \caption{Number of rentals in NYC boroughs vs median price}
  \label{fig: your_label}
\end{figure}

It can be seen that there is the greatest number of Airbnb rentals in Manhattan. Since this borough is the major tourist attraction region in New York, it could be why this borough has the highest median price as compared to other boroughs. If potential traveler wants to save money on rentals, they should avoid Manhattan and Brooklyn because their costs are generally higher. \newline



\begin{figure}[H]
  \centering
  \includegraphics[width=0.4\textwidth]{\imagepath/ishost_superhost.png}
  \caption{Median price for superhosts}
  \label{fig: your_label}
\end{figure}

\begin{figure}[H]
  \centering
  \includegraphics[width=0.4\textwidth]{\imagepath/host_identity_verified.png}
  \caption{Host identity verified vs median price}
  \label{fig: your_label}
\end{figure}

The median rental price hosted by super hosts was slightly higher than that hosted by non-super hosts. Similar results are seen for hosts whose identity has been verified. If a host’s identity is verified, or if a host is an experienced, it will increase confidence for potential tourists to select such rentals, ultimately driving up the price and demand of such rentals. Therefore, such findings suggest keeping the columns for modeling purposes. \newline

\begin{figure}[H]
  \centering
  \includegraphics[width=0.4\textwidth]{\imagepath/property_types_vs_price.png}
  \caption{Property type vs median price}
  \label{fig: your_label}
\end{figure}

The above barplot shows the median prices of different property types across the 5 boroughs in NYC. The hotel price had the highest median price in Manhattan. Similarly, the barn category in Brooklyn had almost \$ 380 median price. The barplot indicates that, on average, apartments and hotels tend to be more expensive as compared to other room types. Furthermore, each rental tends to be cheaper in Bronx and Staten Island whereas the rentals in Brooklyn and Manhattan tend to be much more expensive. It could be due to the fact that these broughs tend to attract more tourists that drives up the demand. \newline




\begin{figure}[H]
  \centering
  \includegraphics[width=0.4\textwidth]{\imagepath/room_type_vs_median_price.png}
  \caption{Room type vs median price}
  \label{fig: your_label}
\end{figure}

The above boxplot shows the price distribution of different room types across the 5 boroughs of NYC. It is clearly seen that the hotel rooms of Manhattan have the most expensive price. Similarly, the price of entire homes and apartments are more expensive in all the boroughs. The boxplot also indicates that the price distribution is very skewed to the right across all room types in all broughs in New York City.  Each room type is the most expensive in Manhattan as compared to other boroughs. Bronx and Staten Island have the least expensive room types, which could be due to the fact that these broughs attract the least number of tourists in general. \newline












\begin{comment} 
#$$$$$$$$$$$$$$$$$$$$$$$$$$$$$$$$$$ comment start #####################################
Before you begin to format your paper, first write and save the content as a 
separate text file. Complete all content and organizational editing before 
formatting. Please note sections \ref{AA}--\ref{SCM} below for more information on 
proofreading, spelling and grammar.

Keep your text and graphic files separate until after the text has been 
formatted and styled. Do not number text heads---{\LaTeX} will do that 
for you.

\subsection{Abbreviations and Acronyms}\label{AA}
Define abbreviations and acronyms the first time they are used in the text, 
even after they have been defined in the abstract. Abbreviations such as 
IEEE, SI, MKS, CGS, ac, dc, and rms do not have to be defined. Do not use 
abbreviations in the title or heads unless they are unavoidable.

\subsection{Units}
\begin{itemize}
\item Use either SI (MKS) or CGS as primary units. (SI units are encouraged.) English units may be used as secondary units (in parentheses). An exception would be the use of English units as identifiers in trade, such as ``3.5-inch disk drive''.
\item Avoid combining SI and CGS units, such as current in amperes and magnetic field in oersteds. This often leads to confusion because equations do not balance dimensionally. If you must use mixed units, clearly state the units for each quantity that you use in an equation.
\item Do not mix complete spellings and abbreviations of units: ``Wb/m\textsuperscript{2}'' or ``webers per square meter'', not ``webers/m\textsuperscript{2}''. Spell out units when they appear in text: ``. . . a few henries'', not ``. . . a few H''.
\item Use a zero before decimal points: ``0.25'', not ``.25''. Use ``cm\textsuperscript{3}'', not ``cc''.)
\end{itemize}

\subsection{Equations}
Number equations consecutively. To make your 
equations more compact, you may use the solidus (~/~), the exp function, or 
appropriate exponents. Italicize Roman symbols for quantities and variables, 
but not Greek symbols. Use a long dash rather than a hyphen for a minus 
sign. Punctuate equations with commas or periods when they are part of a 
sentence, as in:
\begin{equation}
a+b=\gamma\label{eq}
\end{equation}

Be sure that the 
symbols in your equation have been defined before or immediately following 
the equation. Use ``\eqref{eq}'', not ``Eq.~\eqref{eq}'' or ``equation \eqref{eq}'', except at 
the beginning of a sentence: ``Equation \eqref{eq} is . . .''

\subsection{\LaTeX-Specific Advice}

Please use ``soft'' (e.g., \verb|\eqref{Eq}|) cross references instead
of ``hard'' references (e.g., \verb|(1)|). That will make it possible
to combine sections, add equations, or change the order of figures or
citations without having to go through the file line by line.

Please don't use the \verb|{eqnarray}| equation environment. Use
\verb|{align}| or \verb|{IEEEeqnarray}| instead. The \verb|{eqnarray}|
environment leaves unsightly spaces around relation symbols.

Please note that the \verb|{subequations}| environment in {\LaTeX}
will increment the main equation counter even when there are no
equation numbers displayed. If you forget that, you might write an
article in which the equation numbers skip from (17) to (20), causing
the copy editors to wonder if you've discovered a new method of
counting.

{\BibTeX} does not work by magic. It doesn't get the bibliographic
data from thin air but from .bib files. If you use {\BibTeX} to produce a
bibliography you must send the .bib files. 

{\LaTeX} can't read your mind. If you assign the same label to a
subsubsection and a table, you might find that Table I has been cross
referenced as Table IV-B3. 

{\LaTeX} does not have precognitive abilities. If you put a
\verb|\label| command before the command that updates the counter it's
supposed to be using, the label will pick up the last counter to be
cross referenced instead. In particular, a \verb|\label| command
should not go before the caption of a figure or a table.

Do not use \verb|\nonumber| inside the \verb|{array}| environment. It
will not stop equation numbers inside \verb|{array}| (there won't be
any anyway) and it might stop a wanted equation number in the
surrounding equation.

\subsection{Some Common Mistakes}\label{SCM}
\begin{itemize}
\item The word ``data'' is plural, not singular.
\item The subscript for the permeability of vacuum $\mu_{0}$, and other common scientific constants, is zero with subscript formatting, not a lowercase letter ``o''.
\item In American English, commas, semicolons, periods, question and exclamation marks are located within quotation marks only when a complete thought or name is cited, such as a title or full quotation. When quotation marks are used, instead of a bold or italic typeface, to highlight a word or phrase, punctuation should appear outside of the quotation marks. A parenthetical phrase or statement at the end of a sentence is punctuated outside of the closing parenthesis (like this). (A parenthetical sentence is punctuated within the parentheses.)
\item A graph within a graph is an ``inset'', not an ``insert''. The word alternatively is preferred to the word ``alternately'' (unless you really mean something that alternates).
\item Do not use the word ``essentially'' to mean ``approximately'' or ``effectively''.
\item In your paper title, if the words ``that uses'' can accurately replace the word ``using'', capitalize the ``u''; if not, keep using lower-cased.
\item Be aware of the different meanings of the homophones ``affect'' and ``effect'', ``complement'' and ``compliment'', ``discreet'' and ``discrete'', ``principal'' and ``principle''.
\item Do not confuse ``imply'' and ``infer''.
\item The prefix ``non'' is not a word; it should be joined to the word it modifies, usually without a hyphen.
\item There is no period after the ``et'' in the Latin abbreviation ``et al.''.
\item The abbreviation ``i.e.'' means ``that is'', and the abbreviation ``e.g.'' means ``for example''.
\end{itemize}
An excellent style manual for science writers is \cite{b7}.

\subsection{Authors and Affiliations}
\textbf{The class file is designed for, but not limited to, six authors.} A 
minimum of one author is required for all conference articles. Author names 
should be listed starting from left to right and then moving down to the 
next line. This is the author sequence that will be used in future citations 
and by indexing services. Names should not be listed in columns nor group by 
affiliation. Please keep your affiliations as succinct as possible (for 
example, do not differentiate among departments of the same organization).

\subsection{Identify the Headings}
Headings, or heads, are organizational devices that guide the reader through 
your paper. There are two types: component heads and text heads.

Component heads identify the different components of your paper and are not 
topically subordinate to each other. Examples include Acknowledgments and 
References and, for these, the correct style to use is ``Heading 5''. Use 
``figure caption'' for your Figure captions, and ``table head'' for your 
table title. Run-in heads, such as ``Abstract'', will require you to apply a 
style (in this case, italic) in addition to the style provided by the drop 
down menu to differentiate the head from the text.

Text heads organize the topics on a relational, hierarchical basis. For 
example, the paper title is the primary text head because all subsequent 
material relates and elaborates on this one topic. If there are two or more 
sub-topics, the next level head (uppercase Roman numerals) should be used 
and, conversely, if there are not at least two sub-topics, then no subheads 
should be introduced.

\subsection{Figures and Tables}
\paragraph{Positioning Figures and Tables} Place figures and tables at the top and 
bottom of columns. Avoid placing them in the middle of columns. Large 
figures and tables may span across both columns. Figure captions should be 
below the figures; table heads should appear above the tables. Insert 
figures and tables after they are cited in the text. Use the abbreviation 
``Fig.~\ref{fig}'', even at the beginning of a sentence.

\begin{table}[htbp]
\caption{Table Type Styles}
\begin{center}
\begin{tabular}{|c|c|c|c|}
\hline
\textbf{Table}&\multicolumn{3}{|c|}{\textbf{Table Column Head}} \\
\cline{2-4} 
\textbf{Head} & \textbf{\textit{Table column subhead}}& \textbf{\textit{Subhead}}& \textbf{\textit{Subhead}} \\
\hline
copy& More table copy$^{\mathrm{a}}$& &  \\
\hline
\multicolumn{4}{l}{$^{\mathrm{a}}$Sample of a Table footnote.}
\end{tabular}
\label{tab1}
\end{center}
\end{table}

\begin{figure}[htbp]
\centerline{\includegraphics{fig1.png}}
\caption{Example of a figure caption.}
\label{fig}
\end{figure}

Figure Labels: Use 8 point Times New Roman for Figure labels. Use words 
rather than symbols or abbreviations when writing Figure axis labels to 
avoid confusing the reader. As an example, write the quantity 
``Magnetization'', or ``Magnetization, M'', not just ``M''. If including 
units in the label, present them within parentheses. Do not label axes only 
with units. In the example, write ``Magnetization (A/m)'' or ``Magnetization 
\{A[m(1)]\}'', not just ``A/m''. Do not label axes with a ratio of 
quantities and units. For example, write ``Temperature (K)'', not 
``Temperature
#$$$$$$$$$$$$$$$$$$$$$$$$$$$$$$$$$$ comment END #####################################/K''.
\end{comment}
\section*{Acknowledgment}

The preferred spelling of the word ``acknowledgment'' in America is without 
an ``e'' after the ``g''. Avoid the stilted expression ``one of us (R. B. 
G.) thanks $\ldots$''. Instead, try ``R. B. G. thanks$\ldots$''. Put sponsor 
acknowledgments in the unnumbered footnote on the first page.

\section*{References}

Please number citations consecutively within brackets \cite{b1}. The 
sentence punctuation follows the bracket \cite{b2}. Refer simply to the reference 
number, as in \cite{b3}---do not use ``Ref. \cite{b3}'' or ``reference \cite{b3}'' except at 
the beginning of a sentence: ``Reference \cite{b3} was the first $\ldots$''

Number footnotes separately in superscripts. Place the actual footnote at 
the bottom of the column in which it was cited. Do not put footnotes in the 
abstract or reference list. Use letters for table footnotes.

Unless there are six authors or more give all authors' names; do not use 
``et al.''. Papers that have not been published, even if they have been 
submitted for publication, should be cited as ``unpublished'' \cite{b4}. Papers 
that have been accepted for publication should be cited as ``in press'' \cite{b5}. 
Capitalize only the first word in a paper title, except for proper nouns and 
element symbols.

For papers published in translation journals, please give the English 
citation first, followed by the original foreign-language citation \cite{b6}.

\begin{thebibliography}{00}
\bibitem{b1} G. Eason, B. Noble, and I. N. Sneddon, ``On certain integrals of Lipschitz-Hankel type involving products of Bessel functions,'' Phil. Trans. Roy. Soc. London, vol. A247, pp. 529--551, April 1955.
\bibitem{b2} J. Clerk Maxwell, A Treatise on Electricity and Magnetism, 3rd ed., vol. 2. Oxford: Clarendon, 1892, pp.68--73.
\bibitem{b3} I. S. Jacobs and C. P. Bean, ``Fine particles, thin films and exchange anisotropy,'' in Magnetism, vol. III, G. T. Rado and H. Suhl, Eds. New York: Academic, 1963, pp. 271--350.
\bibitem{b4} K. Elissa, ``Title of paper if known,'' unpublished.
\bibitem{b5} R. Nicole, ``Title of paper with only first word capitalized,'' J. Name Stand. Abbrev., in press.
\bibitem{b6} Y. Yorozu, M. Hirano, K. Oka, and Y. Tagawa, ``Electron spectroscopy studies on magneto-optical media and plastic substrate interface,'' IEEE Transl. J. Magn. Japan, vol. 2, pp. 740--741, August 1987 [Digests 9th Annual Conf. Magnetics Japan, p. 301, 1982].
\bibitem{b7} M. Young, The Technical Writer's Handbook. Mill Valley, CA: University Science, 1989.
\end{thebibliography}
\vspace{12pt}
\color{red}
IEEE conference templates contain guidance text for composing and formatting conference papers. Please ensure that all template text is removed from your conference paper prior to submission to the conference. Failure to remove the template text from your paper may result in your paper not being published.

\end{document}
